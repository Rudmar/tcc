\part{Introdução}
\chapter[Introdução]{Introdução}
\addcontentsline{toc}{chapter}{Introdução}

As tecnologias vem se inovando e trazendo novas ferramentas constantemente, o investimeto em novas platarformas é enorme e, como não podia deixar de ser, o resultado é uma crescente evolução da interação com a sociedade de forma geral, especialmente na sociedade de consumo, que por meio destas inovações, cada vez mais encontram novas formas de comprar, comparar e obter orçamentos de produtos e serviços, onde um consumidor busca a melhor oferta para seu perfil ou necessidade. Da mesma forma, o cenário também muda para quem vende, usando destes artifícios para oferecer o que se encaixa à seus clientes, no melhor momento e a entender suas necessidades, obtendo maior eficácia nas vendas.

Desde os primórdios das civilizações, já existem notícias de que haviam trocas de itens e matéria-prima, os primeiros povos a desenvolverem esta prática foram os Fenícios, Árabes, Assírios e Babilônios, por meio da comunicação, datando o comério como tão velho quanto as primeiras organizações humanas. As embarcações que chegaram ao Brasil, saíram de Portugal com um objetivo principal de encontrar as chamadas especiarias, as quais eram o principal item comercial da época\cite{furtado:2009}.

Com o passar do tempo, podemos ver uma migração do comércio para o ambiente virtual, denominado como comércio eletronico ou \textit{E-Commerce} qualquer empresa que não esteja incluida nesta vertente digital acaba sendo defasada no mercado financeiro, ficando atrás de suas concorrentes\cite{roque:2012}, por isso, a crescente no mercado do \textit{E-Commerce}, grandes nomes vem se adaptando aos novos paradigmas de compra e venda, buscando apliar seu mercado e alcançar seus clientes, visto que um comércio eletrônico funciona 24 horas por dia, não encontra fronteiras em suas vendas, é confortável para o consumidor, entre outras vantagens, encontradas no artigo blog Comércio Eletrônico\footnote{\url{http://www.comercioeletronico.blog.br/2009/01/19/dez-motivos-para-investir-do-comercio-eletronico/}}, esta abordagem se mostra extremamente útil para o contexto.

Desde sua criação, em 1980, a \textit{Web} já tem muita informação usada ativamente, mas apesar de suas drásticas transformações através do tempo, ainda hoje consultas devem ser feitas de maneira exaustiva através da \textit{Web}. Com o intenção de ajudar usário nas buscas e oferecer solução para o "caos informacional", foi proposta a \textit{Web Semântica}, seu objetivo principal é tornar a informação legível para máquinas de maneira que humanos e computadores ou \textit{end-points} trabalharão em cooperação, onde temos um dado com um "significado bem definido"\cite{berners:2001}.

Ontologia vem do estudo Filosófico que trata da natureza do ser, do real, do existir entre entes e questões gerais de metafísicas. A morfologia da palavra nos remete a \textit{ontos+logoi} = "conhecimento do ser". Para a área da computação, uma ontologia é uma especificação formal e explícita de um conceito compartilhado. Por conceito compartilhado lê-se, objetos, entidades e relacinamentos de certo domínio, bem como características que podem ser relevantes para tal contexto\cite{gruber:1992}. 

Para o contexto da pesquisa deste trabalho teremos uma aplicação \textit{Web} \textit{E-Commerce} para venda de produtos de restaurante, portanto a ontologia a ser construída(ou melhorada?) será voltada para este ambiente, assim como a elicitação dos requisitos por meio dos clientes, resultados e metodologias, estas que serão abordadas nos capítulos próximos.

\section{Contexto}

Após um primeiro contato com uma empresa de consultoria e distribuição de artigos de hotelaria, mesa e bar, com a intenção de construir um \textit{E-Commerce} para seu negócio, foi abordada a ideia de montar este sistema utilizando \textit{Web Semântica}, proposta que foi bem aceita pelos \textit{stakeholders} da empresa. Acrescida a oportunidade, viu-se a chance de realizar a pesquisa e desenvolvimento de uma ontologia que possa dar apoio a construção da aplicação. A empresa trabalha com um catálogo que conta com mais de 8.000 produtos, e utilizam uma plataforma \textit{on-line} onde um cliente pode fazer um orçamento, informando os produtos desejados e a quantidade, porém, isso gera um pedido que deve ser impresso e informado a um vendedor, para que este entre em contato com o cliente para que a venda possa ser efetivada, segundo o próprio cliente, este sistema encontra-se extremamente defasado e por isso, estariam perdendo oportunidades de venda.

Por seu porte e renome, a empresa atende pedidos de larga escala em várias partes do Brasil, por isso, um \textit{E-Commerce} aumentaria a responsividade no atendimento a clientes onde não existem lojas físicas próximas\footnote{\url{http://www.comercioeletronico.blog.br/2009/01/19/dez-motivos-para-investir-do-comercio-eletronico/}}, sendo assim, este trabalho utilizará tecnicas de Engenharia de Software para propor uma base para criação da ontologia necessária a este desenvolvimento.

A ideia \textit{core} do produto a ser desenvolvido é uma aplicação em que o cliente terá visibilidade dos itens do catálogo, bem como especificações, preço e características, existindo o interesse por parte do cliente, o mesmo poderá adicionar os produtos desejados em uma espécie de carrinho de compras, tendo a disponibilidade de definir quantidade e se deseja realmente aquele produto no orçamento final, tendo decidido pelos itens e quantidade, o usuário poderá finalizar a compra, inserindo um cartão ou gerando um boleto para efetuar o pagamento e seu pedido já seria automaticamente efetivado, não necessitando de funcionário para aprovar o pedido. A proposta da \textit{Web Semântica} neste sistema seria que, ao adicionar um tipo de produto específico ao seu carrinho de compras, os produtos a serem propostos para o usuário teriam ligação com o produto já escolhido, utilizando da ideia de \textit{linked data}, que é, em suma, usar a \textit{Web} para criar conexões entre dados de dois diferentes recursos\footnote{\url{https://www.w3.org/DesignIssues/LinkedData.html}}. Gerada essa conexão entre itens de catálogo e conjunto de itens que o usuário escolheu, o sistema poderia sugerir com maior eficácia os próximos itens a serem adicionados ao carrinho de compras.

\section{Problema}

Atualmente o \textit{site} que a empresa tem em seu domínio é confuso e mal elaborado, com um \textit{desing} antiquado, que tende a afastar e desanimar o cliente, e tem a única funcionalidade de elaborar orçamentos, sem qualquer inteligência, sendo que, segundo a empresa, os pedidos que chegam por meio da \textit{web} são raros e na maioria das vezes nem acontecem. Tendo em vista que as opções mais comuns para atingir clientes é baseada no uso de cookies, arquivo o qual navegadores guardam informações trocadas entre o próprio navegador e o servidor da página visitada. Partindo daí e agregando técnicas estatísticas, de aquisição de conhecimento e aprendizado de máquina, pode-se gerar dados os quais a empresa pode utilizar para alcançar de maneira mais eficiente seu cliente, com um produto ou serviço específico\cite{boland:2014}.

Um dos maiores desafios da questão de publicidade contexto de consumo atual é a organização e interpretação dos dados coletados, sabidamente bastantes valiosos, mas que não são utilizados em sua total capacidade de aproveitamento\cite{zamanzadeh:2013}.

Esse conjunto de causas geram problemas que são comuns na maioria dos casos em que se aplicam essas técnicas e tecnologias.

\begin{itemize}
\item Atraso de atendimento de necessidades
\item Imprecisão de sugestões de compra
\item Generalização de dados
\item Desistência de compras
\item Mal uso de recursos
\end{itemize}

Com estes problemas, que são recorrentes em vários tipos de modelos de negócios atuantes na ramo virtual, podemos verificar a ineficácia da abordagem adotada, vezes por falta de informação, outras por falta de competência de quem constrói tais plataformas.

	Dado tais problemas, este trabalho visa responder as seguintes questões baseadas no contexto em que se aplica:

	\begin{itemize}
	\item{\textit{Qual a visão de valor agregado para um cliente que deseja a tecnologia Web Semântica em sua aplicação?}}
	\item{\textit{Qual a melhor ontologia a ser aplicada para um sistema de E-Commerce?}}
	\end{itemize}

	Para assim propor uma solução que responda o seguinte questionamento:
	\begin{itemize}
	\item{\textit{Como oferecer uma plataforma que atenda as necessidades de um cliente e melhorar a eficácia da sugestão de itens a serem comprados em um E-Commerce?}}
	\end{itemize}

\section{Objetivos}

Para este trabalho, tem-se como objetivo especificar os requisitos básicos de uma plataforma \textit{E-Commerce} para um modelo de negócio específico, no caso, uma empresa de consultoria e distribuição de artigos de hotelaria, mesa e bar somadas a uma documentação de riscos e uma base ontológica para o sistema. Para isto, foram definidos os seguintes objetivos específicos:

	\begin{itemize}
	\item{Apresentar o estado da arte as quais utilizam as tecnologias de \textit{Web Semântica}}
	\item{Elicitar requisitos funcionais para uma aplicação \textit{E-Commerce}}
	\item{Realizar uma modelagem conceitual e propor uma ontologia no contexto citado para o modelo de negócio da empresa}
	\end{itemize}

\section{Metodologia}

Afim de atacar os objetivos específicos citados, foram escolhidas metodologias que ajudarão no melhor entendimento do problema, de maneira a contribuir com uma possível solução.

	\begin{enumerate}
	\item{Pesquisas bibliográficas com foco nos problemas e em soluções, assim como verificar métodos que auxiliem na costrução de soluções. A pesquisa bibliográfica é uma abordagem técnica que visa coletar informações materiais já publicadas em livros, jornais, revistas, artigos ou qualquer outro tipo de material de acesso possível\cite{gil:2008}}

		\begin{quote} A pesquisa bibliográfica é fetia a partir do levantamento de referências teóricas já analisadas, e publicadas por meios escritos ou eletrônicos, como livros, artigos científicos, páginas de \textit{Web Sites}. Todo trabalho científico inicia-se com pesquisa 				bibliográfica, que permite ao pesquisador conhecer o que já foi estudado sobre o assunto. Existem também pesquisas científicas que se baseinham única e exclusivamente na pesquisa bibliográfica, procurando referências teórias publicadas com o objetivo de 				coletar informações e conhecimentos prévios sobre um tema a respeito do qual se busca respostas\cite{fonseca:2002}.
		\end{quote}
	\item{Com o apoio das pesquisas bibliográficas, será realizada estudo de abordagens práticas do uso de ontologias, com foco principal no \textit{Framework} denominado \textit{GoodRelations} que é a principal ferramenta de ontologia voltada para \textit{E-Commerce}, descrevendo, analisando e avaliando de acordo com critérios específicos, definindo uma ferramenta para coletar dados a serem colecionados. Tal método de pesquisa e análise é chamado exploratório e descritivo, tem como objetivo especificar as características das ontologias \cite{gil:2008}}
	\item{Aplicando técnicas de elicitação e validação de será feito um levantamento de requisitos e necessidades junto aos \textit{Stakeholders} da empresa, com isso, gerar protótipos iterativamente, para auxiliar no desenvolvimento da solução proposta. A partir disso, definir as características principais priorizadas pela empresa, de maneira a aplicar o conceito inicial de \textit{Web Semântica} e validar com \textit{Product Owners}. Este levantamento terá suporte da empresa, esta assumindo o papel de um caso a ser estudado, Estudo de Caso que tem como objetivo compreender o evento em estudo e desenvolver proposições a respeito do objeto\cite{gil:2008}.}
	\item{Partindo das pesquisas bibliográficas, análises de soluções existentes e do levantamento de requisitos, apoiado pela metodologia 101 de desenvolvimento, será proposta uma modelagem conceitual de ontologia. Tal metodologia é um processo que visa facilitar a criação de ontologias. Consiste em um guia de passos iterativos a serem executados para desenvolver ontologias\cite{rautenberg:2010}. Utilizando as ferramentas ASTAH\footnote{url{https://astah.com}} e Protégé\footnote{url{http://protege.stanford.edu}} que vão dar base para a concepção, modelagem e construção da ontologia.}
	\item{Com o escopo definido e as necessidades mapeadas pelos requisitos do sistema, a solução será desenhada, utilizando técnicas de desenvolvimento de software e modelagem de processos. O padrão a ser utilizado na modelagem é o UML\footnote{url{www.uml.org}}, para construir documentos e planejar. As fases do projeto serão por iterações icrementais.}

	\end{enumerate}

Para o Trabalho de Conclusão de Curso 1, tem-se como escopo atingir os objetivos citados no tópico 1, estes serão dcumentados e analisados por meio das metodologias propostas. Os objetivos citados nos tópicos 2, 3, 4 e 5 serão abordadas no Trabalho de Conclusão de Curso 2, da mesma forma documentados e analisados, evoluindo também o tópico 1.

\section{Organização do trabalho}
(...)





