\part{Introdução}
\chapter[Introdução]{Introdução}
\addcontentsline{toc}{chapter}{Introdução}

As tecnologias vem se inovando e trazendo novas ferramentas constantemente, o investimeto em novas platarformas é enorme e, como não podia deixar de ser, o resultado é uma crescente evolução da interação com a sociedade de forma geral, especialmente na sociedade de consumo, que por meio destas inovações, cada vez mais encontram novas formas de comprar, comparar e obter orçamentos de produtos e serviços, onde um consumidor busca a melhor oferta para seu perfil ou necessidade. Da mesma forma, o cenário também muda para quem vende, usando destes artifícios para oferecer o que se encaixa à seus clientes, no melhor momento e a entender suas necessidades, obtendo maior eficácia nas vendas.

Desde os primórdios das civilizações, já existem notícias de que haviam trocas de itens e matéria-prima, os primeiros povos a desenvolverem esta prática foram os Fenícios, Árabes, Assírios e Babilônios, por meio da comunicação, datando o comério como tão velho quanto as primeiras organizações humanas. As embarcações que chegaram ao Brasil, saíram de Portugal com um objetivo principal de encontrar as chamadas especiarias, as quais eram o principal item comercial da época\cite{furtado:2009}.

Com o passar do tempo, podemos ver uma migração do comércio para o ambiente virtual, denominado como comércio eletronico ou \textit{E-Commerce} qualquer empresa que não esteja incluida nesta vertente digital acaba sendo defasada no mercado financeiro, ficando atrás de suas concorrentes\cite{roque:2012}, por isso, a crescente no mercado do \textit{E-Commerce}, grandes nomes vem se adaptando aos novos paradigmas de compra e venda, buscando apliar seu mercado e alcançar seus clientes, visto que um comércio eletrônico funciona 24 horas por dia, não encontra fronteiras em suas vendas, é confortável para o consumidor, entre outras vantagens, encontradas no artigo blog Comércio Eletrônico\footnote{\url{http://www.comercioeletronico.blog.br/2009/01/19/dez-motivos-para-investir-do-comercio-eletronico/}}, esta abordagem se mostra extremamente útil para o contexto.

Desde sua criação, em 1980, a \textit{Web} já tem muita informação usada ativamente, mas apesar de suas drásticas transformações através do tempo, ainda hoje consultas devem ser feitas de maneira exaustiva através da \textit{Web}. Com o intenção de ajudar usário nas buscas e oferecer solução para o "caos informacional", foi proposta a \textit{Web Semântica}, seu objetivo principal é tornar a informação legível para máquinas de maneira que humanos e computadores ou \textit{end-points} trabalharão em cooperação, onde temos um dado com um "significado bem definido"\cite{berners:2001}.

Ontologia vem do estudo Filosófico que trata da natureza do ser, do real, do existir entre entes e questões gerais de metafísicas. A morfologia da palavra nos remete a \textit{ontos+logoi} = "conhecimento do ser". Para a área da computação, uma ontologia é uma especificação formal e explícita de um conceito compartilhado. Por conceito compartilhado lê-se, objetos, entidades e relacinamentos de certo domínio, bem como características que podem ser relevantes para tal contexto\cite{gruber:1992}. 

Para o contexto da pesquisa deste trabalho teremos uma aplicação \textit{Web} \textit{E-Commerce} para venda de produtos de restaurante, portanto a ontologia a ser construída(ou melhorada?) será voltada para este ambiente, assim como a elicitação dos requisitos por meio dos clientes, resultados e metodologias, estas que serão abordadas nos capítulos próximos.