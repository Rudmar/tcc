\part{Introdução}
\chapter[Introdução]{Introdução}
\addcontentsline{toc}{chapter}{Introdução}

Aqui abordaremos como as tecnologias vem se inovando e trazendo novas ferramentas constantemente, o investimeto em novas platarformas é enorme e, como não podia deixar de ser, o resultado é uma crescente evolução da interação com a sociedade de forma geral, especialmente na sociedade de consumo, que por meio destas inovações, cada vez mais encontram novas formas de comprar, comparar e obter orçamentos de produtos e serviços, onde um consumidor busca a melhor oferta para seu perfil ou necessidade. Da mesma forma, o cenário também muda para quem vende, usando destes artifícios para oferecer o que se encaixa à seus clientes, no melhor momento e a entender suas necessidades, obtendo maior eficácia nas vendas.

Este parágrafo fala a respeito do comércio, desde os primórdios das civilizações, já existem notícias de que haviam trocas de itens e matéria-prima, os primeiros povos a desenvolverem esta prática foram os Fenícios, Árabes, Assírios e Babilônios, por meio da comunicação, datando o comério como tão velho quanto as primeiras organizações humanas. As embarcações que chegaram ao Brasil, saíram de Portugal com um objetivo principal de encontrar as chamadas especiarias, as quais eram o principal item comercial da época\cite{furtado:2009}.

Aqui será tratado a evolução do comércio até os dias de hoje, com o passar do tempo, podemos ver uma migração do comércio para o ambiente virtual, denominado como comércio eletronico ou \textit{E-Commerce} qualquer empresa que não esteja incluida nesta vertente digital acaba sendo defasada no mercado financeiro, ficando atrás de suas concorrentes\cite{roque:2012}, por isso, a crescente no mercado do \textit{E-Commerce}, grandes nomes vem se adaptando aos novos paradigmas de compra e venda, buscando apliar seu mercado e alcançar seus clientes, visto que um comércio eletrônico funciona 24 horas por dia, não encontra fronteiras em suas vendas, é confortável para o consumidor, entre outras vantagens, encontradas no artigo blog Comércio Eletrônico\footnote{\url{http://www.comercioeletronico.blog.br/2009/01/19/dez-motivos-para-investir-do-comercio-eletronico/}}, esta abordagem se mostra extremamente útil para o contexto.

Trata-se aqui a respeito de \textit{Web Semântica}, desde sua criação, em 1980, a \textit{Web} já tem muita informação usada ativamente, mas apesar de suas drásticas transformações através do tempo, ainda hoje consultas devem ser feitas de maneira exaustiva através da \textit{Web}. Com o intenção de ajudar usário nas buscas e oferecer solução para o "caos informacional", foi proposta a \textit{Web Semântica}, seu objetivo principal é tornar a informação legível para máquinas de maneira que humanos e computadores ou \textit{end-points} trabalharão em cooperação, onde temos um dado com um "significado bem definido"\cite{berners:2001}.

A respeito de Ontologias, sabemos que o conceito vem do estudo Filosófico que trata da natureza do ser, do real, do existir entre entes e questões gerais de metafísicas. A morfologia da palavra nos remete a \textit{ontos+logoi} = "conhecimento do ser". Para a área da computação, uma ontologia é uma especificação formal e explícita de um conceito compartilhado. Por conceito compartilhado lê-se, objetos, entidades e relacinamentos de certo domínio, bem como características que podem ser relevantes para tal contexto\cite{gruber:1992}. 

\section{Contexto}

Aqui fala-se do contexto atual do problema a ser abordada no artigo, o \textit{E-Commerce} vai na contra-mão da economia brasileira para o ano de 2016-2017, que tem perspectiva de crescimento de 25\% para o ano de 2017, percebe-se o comércio eletrônico amadurecendo e obtendo destaque em coorporações, seja pelo processo inevitável de mudança da forma como o consumidor compra, ou por conta das oportunidades de negócios\cite{roque:2012}. A tendência do \textit{E-Commerce} é continuar crescendo, segundo pesquisas, a estimativa para 2021 é de arrecadar o montante de R\$229,53 bilhões.\footnote{\url{http://www.conversion.com.br/blog/conversion-lanca-relatorio-inedito-sobre-o-e-commerce-no-brasil/}}

O \textit{E-Commerce} brasileiro arrecada aproximadamente 55,81 bilhões/R\$, mostrando um crescimento de 29,6\% comparado ao ano de 2014, por isso, se mostra uma plataforma de gigante potencial, porém ainda pouco explorado, tem-se um grande número de informações a serem trabalhadas e conectadas a respeito de usuários e a nível de mercado, por isso, esse tipo de dados é muito bem aproveitado com o conceito de \textit{linked data}\cite{berners:2001}. Com a proposta de estabelecer ordem ao caos da informação, muitas plataformas de \textit{E-Commerce} já utilizam o apoio da \textit{Web Semântica} para atingir seus clientes, percebendo a maior eficiência de vendas em produtos sugeridos utilizando uma ontologia, como exemplo, podemos citar a gigante \textit{Best Buy}
que utiliza a ontologia do \textit{framework GoodRelations}. Porém, esse conjunto de conceitos a aplicações ainda é pouco utilizado no Brasil, visto a baixa taxa de conversão dos \textit{E-Commerce} brasileiro, sendo de 1,04\% em 2016. Essa baixa taxa de conversão, demonstra a defasagem do mercado brasileiro quanto ao mercado exterior, um \textit{E-Commerce} aumentaria a responsividade no atendimento a clientes onde não existem lojas físicas próximas entre várias outros benefícios\footnote{\url{http://www.comercioeletronico.blog.br/2009/01/19/dez-motivos-para-investir-do-comercio-eletronico/}}

\begin{quote}''Taxa de conversão é a proporção entre visitas a sites de
e-commerce e número de pedidos realizados pelos
consumidores. Por exemplo, se a taxa de conversão for de
2\% significa que a cada 100 visitas um site realizou 2
vendas\cite{convension:2016}.''
\end{quote}

\section{Problema}

Tendo em vista que as opções mais comuns para atingir clientes é baseada no uso de cookies, arquivo o qual navegadores guardam informações trocadas entre o próprio navegador e o servidor da página visitada. Partindo daí e agregando técnicas estatísticas, de aquisição de conhecimento e aprendizado de máquina, pode-se gerar dados os quais a empresa pode utilizar para alcançar de maneira mais eficiente seu cliente, com um produto ou serviço específico\cite{boland:2014}.

Um dos maiores desafios da questão de publicidade contexto de consumo atual é a organização e interpretação dos dados coletados, sabidamente bastantes valiosos, mas que não são utilizados em sua total capacidade de aproveitamento\cite{zamanzadeh:2013}.

Esse conjunto de causas geram problemas que são comuns na maioria dos casos em que se aplicam essas técnicas e tecnologias.

\begin{itemize}
\item Defasagem no mercado comercial: O mercado atual, principalmente no exterior, existem casos em que se tem \textit{E-Commerce} aplicando \textit{Web Semântica}, para o mercado brasileiro, isto evidencia o atraso quando comparado a outros países.
\item Perda de oportunidades de vendas: Utilizando \textit{Web Semântica} aumenta-se eficácia de vendas, por conta das sugestões de produtos baseadas em ontologia, não em produtos da mesma linha ou características.
\item Imprecisão de sugestões produtos a serem comprados: A maior parte das plataformas de venda brasileiras baseia a proposta de produtos a serem comprados por meio do tipo de pesquisa, utilizando \textit{cookies} para tal, trabalhando a semântica, os produtos propostos poderiam atingir o cliente de maneira mais eficaz, gerando maior oportunidade de vendas.
\item Mal uso de recursos: Muitas das empresas brasileiras hospedam páginas na \textit{Web}, porém essas funcionam como um portfólio da empresa, não agregando grande valor a quem tem acesso, sendo esse gasto com hospedagem contínuo, existe a possibilidade de aumentar o valor deste recurso.
\item Mal uso de dados: Visto a grande circulação de usuários em plataformas \textit{E-Commerce} pouco se utilizam os dados coletados para benefício da empresa, seja para propagandas, acompanhamento do cliente ou entendimento de perfil de usuário.
\end{itemize}

Com estes problemas, que são recorrentes em vários tipos de modelos de negócios atuantes na ramo virtual, podemos verificar a ineficácia da abordagem adotada, vezes por falta de informação, outras por falta de competência de quem constrói tais plataformas.

	Dado tais problemas, este trabalho visa responder as seguintes questões baseadas no contexto em que se aplica:

	\begin{itemize}
	\item{\textit{Como a Web Semântica pode ajudar a melhorar a taxa de conversão de uma plataforma E-Commerce?}}
	\item{\textit{Utilizando a metodologia 101 de ontologia podemos gerar uma ontologia para um caso específico de E-Commerce?}}
	\end{itemize}

	Para assim propor uma solução que responda o seguinte questionamento:
	\begin{itemize}
	\item{\textit{Quais vantagens de um E-Commerce com inteligência Web Semântica para um negócio?}}
	\end{itemize}

\section{Objetivos}

Para este trabalho, tem-se como objetivo especificar os requisitos básicos de uma plataforma \textit{E-Commerce} para um modelo de negócio específico, no caso, uma empresa de consultoria e distribuição de artigos de hotelaria, mesa e bar somadas a uma documentação de riscos e uma base ontológica para o sistema. Para isto, foram definidos os seguintes objetivos específicos:

	\begin{itemize}
	\item{Apresentar o estado da arte as quais utilizam as tecnologias de \textit{Web Semântica}}
	\item{Elicitar requisitos funcionais para uma aplicação \textit{E-Commerce}}
	\item{Realizar uma modelagem conceitual e propor uma ontologia no contexto citado para o modelo de negócio da empresa}
	\end{itemize}

\section{Metodologia}

Afim de atacar os objetivos específicos citados, foram escolhidas metodologias que ajudarão no melhor entendimento do problema, de maneira a contribuir com uma possível solução.

	\begin{enumerate}
	\item{Pesquisas bibliográficas com foco nos problemas e em soluções, assim como verificar métodos que auxiliem na costrução de soluções. A pesquisa bibliográfica é uma abordagem técnica que visa coletar informações materiais já publicadas em livros, jornais, revistas, artigos ou qualquer outro tipo de material de acesso possível\cite{gil:2008}}

		\begin{quote} A pesquisa bibliográfica é feita a partir do levantamento de referências teóricas já analisadas, e publicadas por meios escritos ou eletrônicos, como livros, artigos científicos, páginas de \textit{Web Sites}. Todo trabalho científico inicia-se com pesquisa 				bibliográfica, que permite ao pesquisador conhecer o que já foi estudado sobre o assunto. Existem também pesquisas científicas que se baseinham única e exclusivamente na pesquisa bibliográfica, procurando referências teórias publicadas com o objetivo de 				coletar informações e conhecimentos prévios sobre um tema a respeito do qual se busca respostas\cite{fonseca:2002}.
		\end{quote}
	\item{Com o apoio das pesquisas bibliográficas, será realizada estudo de abordagens práticas do uso de ontologias, com foco principal no \textit{Framework} denominado \textit{GoodRelations} que é a principal ferramenta de ontologia voltada para \textit{E-Commerce}, descrevendo, analisando e avaliando de acordo com critérios específicos, definindo uma ferramenta para coletar dados a serem colecionados. Tal método de pesquisa e análise é chamado exploratório e descritivo, tem como objetivo especificar as características das ontologias \cite{gil:2008}}
	\item{Aplicando técnicas de elicitação e validação de será feito um levantamento de requisitos e necessidades junto aos \textit{Stakeholders} da empresa, com isso, gerar protótipos iterativamente, para auxiliar no desenvolvimento da solução proposta. A partir disso, definir as características principais priorizadas pela empresa, de maneira a aplicar o conceito inicial de \textit{Web Semântica} e validar com \textit{Product Owners}. Este levantamento terá suporte da empresa, esta assumindo o papel de um caso a ser estudado, Estudo de Caso que tem como objetivo compreender o evento em estudo e desenvolver proposições a respeito do objeto\cite{gil:2008}.}
	\item{Partindo das pesquisas bibliográficas, análises de soluções existentes e do levantamento de requisitos, apoiado pela metodologia 101 de desenvolvimento, será proposta uma modelagem conceitual de ontologia. Tal metodologia é um processo que visa facilitar a criação de ontologias. Consiste em um guia de passos iterativos a serem executados para desenvolver ontologias\cite{rautenberg:2010}. Utilizando as ferramentas ASTAH\footnote{url{https://astah.com}} e Protégé\footnote{url{http://protege.stanford.edu}} que vão dar base para a concepção, modelagem e construção da ontologia.}
	\item{Com o escopo definido e as necessidades mapeadas pelos requisitos do sistema, a solução será desenhada, utilizando técnicas de desenvolvimento de software e modelagem de processos. O padrão a ser utilizado na modelagem é o UML\footnote{\url{www.uml.org}}, para construir documentos e planejar. As fases do projeto serão por iterações icrementais.}

	\end{enumerate}

Para o Trabalho de Conclusão de Curso 1, tem-se como escopo atingir os objetivos citados no tópico 1, estes serão dcumentados e analisados por meio das metodologias propostas. Os objetivos citados nos tópicos 2, 3, 4 e 5 serão abordadas no Trabalho de Conclusão de Curso 2, da mesma forma documentados e analisados, evoluindo também o tópico 1.

\section{Organização do trabalho}
(...)





