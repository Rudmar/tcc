\begin{resumo}

Com a dinâmica evolução das tecnologias, o mercado, de maneira geral,
 enxerga as novas plataformas como aliadas para alcançar seus clientes com maior
 eficácia na oferta de produtos e serviços. Hoje a tecnologia \textit{on line}
 tem-se mostrado um grande catálogo de informações, nos vemos cada vez
 mais utilizando aplicações que automatizam nosso dia-a-dia, sistemas de gps
 que geram a rota provável assim que deixamos o lar, lojas digitais que filtram
 ofertas pela sua ultima busca, entre muitos outros exemplos. Este trabalho tem
 como objetivo estudar o impacto dessa inteligência para o cliente
 analisando a elicitação de requisitos sob a visão da Engenharia de 
 Software, mediante pesquisas  bibliográficas e entrevistas com clientes  reais, utilizadas para apoiar
 os estudos e conceitos apresentados. Ao final deste trabalho espera-se ter maior
 entendimento de como a Web Semântica pode apoiar e agregar valor a negócios 
 por meio de documentos de elicitação e o impacto de tais funcionalidades sob o ponto de vista
 da Engenharia de Software assim com uma proposta de ontologia para
 um \textit{E-Commerce} para uma empresa de consultoria e distribuição de artigos de hotelaria, mesa e bar.

 \vspace{\onelineskip}
    
 \noindent
 \textbf{Palavras-chaves}: e-commerce; ontologia; mercado; web semântica.
\end{resumo}
