\part{Desenvolvimento}

\chapter[Comércio]{Comércio}
\section{Considerações iniciais}

O termo comércio deriva do conceito latim \textit{comercium} e refere-se à negociação que tem lugar na hora de comprar ou vender serviços ou mercadorias. Também pode-se chamar de comércio qualquer loja, armazém, estabelecimento ou plataforma digital que realize transações de algo em troca de outra coisa de igual valor, podendo ou não ser dinheiro. Estruturou-se como uma atividade socio-econômica que consiste na compra e venda de bens, serviços ou produtos\footnote{\url{conceito.de/comercio}}. Atualmente existem várias classes de comércio, neste documento será abordado o \textit{E-Commerce}, também conhecido como comércio eletrônico.

\section{Modelos de comércio}

O comércio atual é dividido em duas principais categorias, o comércio físico e o comércio eletrônico, para vias de pesquisa, vamos tratar do comércio eletrônico, aquele em que uma transação pode ser iniciada e finalizada por meio de uma plataforma ou aplicação que tenha conexão a \textit{Web}\footnote{\url{http://www.scielo.br/pdf/rae/v42n3/v42n3a10}}.

\subsection{Comércio eletrônico}

Nesta seção vamos tratar a questão do comércio eletrônico atual, que já é uma realidade nos mais diversos setores da economia, tanto falando de Brasil, como também fora dele. Várias organizações tem o \textit{E-Commerce} como parte de sua estratégia, outras já funcionam exclusivamente a partir do comércio eletrônico. Segundo \cite{rae:2002} tanto no mundo como no Brasil, o comércio eletrônico se encontra em processo de consolidação e a tendência é de crescimento. As realização que estão sendo empreendidas tem como foco principal desenvolver os processos que já existem e criar uma base para que ambientes mais novos possam ser sustentados. A partir de um início simples de fornecimento de informação, percebeu-se o potencial de mercado, sendo asim, buscou-se primeiro a realização de transações, em segundo, formas de apoiar à distribuição, e por ultimo a iteração pela comunicação e pela troca de informações\cite{rae:2002}.

\section{Métodos e Tecnologias}

Será abordado neste tópico as metodologias aplicadas no conceito de comércio eletrônico, a pesquisa no mercado brasileiro fornece subsídios suficientes para confirmar que as empresas estão adequando-se aos novos ambientes, melhorando os processos já existentes e utilizando novas tecnologias já usadas no mercado, principalmente \textit{cookies}, esta situação demostra que as empresas estão se movimentando para que possam evoluir junto a esse tipo de comércio\footnote{\url{http://www.dataversity.net/semantic-commerce-structuring-your-retail-website-for-the-next-generation-web-2/}}

\section{Considerações preliminares}

O comércio acompanha a evolução da sociedade, tendo carater de grande importância no quesito sócio-econômico de países, o comércio interno é aquele que se realiza dentro de um mesmo país, que respondem a uma mesma jurisdição, já o externo diz respeito a transações entre diferentes países. É perceptível um grande crescimento do valor do comércio eletrônico para empresas, por isso esse aumento exponencial da busca por plataformas e tecnologias para aplicar \textit{E-Commerce}.

\chapter{Web semântica e Ontologias}
\section{Considerações iniciais}

Neste capítulo tem-se por objetivo revisar bibliográficamente e introduzir conceitos de \textit{Web Semântica}. Abordando os conceitos, tecnologias e algumas aplicações, assim como levantar o uso de \textit{Web Semântica} em exemplos práticos aplicados a \textit{E-Commerce}.

\section{Web Semântica e Ontologias}

Hoje, com o exponente avanço das tecnologias, que geram mudanças de paradigmas e a maneira com que nos relacionamos e comunicamos, o modo como geramos e recebemos informações caminha junto a esta evolução. A chamada "era da informação" ou "era do conhecimento", as informações geradas e consumidas são vistas como matéria-prima para
 desenvolvimento sócio-cultural e econômico.\cite{takahashi:2000}

A principal destas tecnologias criadas para disseminação de informação é a intenet, através da \textit{World Wide Web} (WWW), considerada a maior fonte de informações concentradas\cite{alves:2004}. Em contrapartida a essa excessiva criação de conteúdo, existe também uma crescente quantidade de informações que são disponibilizadas na rede, gerando problemas de busca e recuperação de conteúdo, por conta de uma falta de organização, mesmo com fortes ferramentas de busca\cite{breitman:2006}

A respeito de busca e recuperação de informações a forma que são disponibilizadas na \textit{web} é utilizando linguagens de marcação, tais quais HTML e XML, onde são configuradas suas propriedades\cite{breitman:2006}. Desta maneira, as funcionalidades de busca realizam sua trabalho por meio de similaridades sintáticas, ou seja, algoritmos que buscam palavras-chave, o que por várias vezes resultam em resultados ineficiêntes ou indesejados, isso por uma palavra ter o poder de assumir diferentes significados, dependendo do contexto a qual está inserida\cite{breitman:2006}.

Sendo assim, em meio a esse "caos" de informações, não existe estratégia que abrange a indexação dos documentos contidos na mesma de maneira satisfatória, recuperar estas informações, por meio de motores de busca, baseise prioritáriamente em palavras-chaves contidas no texto dos documentos\cite{souza:2004}

Deste modo, as informações disponibilizadas nada mais são que palavras encontradas em um texto, os computadores tratam de apresenta-las, porém, sem avaliar, classificar ou selecionar essas informações, responsabilidades estas que ficam a cargo dos seres humanos interessados para definir quais informações são condizentes e relevantes a sua busca\cite{breitman:2006}.

Segundo Souza(2006) por mais que a \textit{web} tenha sido, em seu conceito, projetada com a ideia de possibilitar o fácil acesso e agilizar troca de informações, sua implementação ocorreu de forma descentralizada, crescendo exponencialmente e hoje atua como um imenso repositório de documentos, o que deixa a desejar ao que se trata de conceitos relevantes ao conteúdo.

No contraponto ao que é conhecido como \textit{Web Sintática}\cite{berners:2001} a proposta é um mecanismo que capture o significado das palavras de maneira semântica, de forma que os computadores possam processar e relacionar informações capturadas em diferentes fontes e contextos. Baseado nisso, foi proposto a inserção de semântica na estrutura dos documentos disponibilizados via \textit{web}\cite{breitman:2006}. Esta nova arquitetura foi denominada \textit{Web Semântica}, ideia creditada a Tim Berners-Lee e conduzida posteriormente pela W3C, onde o objetivo é acoplar contexto e inteligência na \textit{web} atual, possibilitando melhor uso e recuperação de informações relevantes\cite{souza:2004}.

\begin{quote}
A \textit{Web Semântica} não é uma web separada, mas uma extensão da \textit{web} atual na qual as informações apresentam significados bem definidos e permite que computadores e pessoas possam trabalhar em cooperação\cite{berners:2001}.
\end{quote}

Desta forma, computadores deixam de ser apenas apresentadores de conteúdo e se tornam agentes inteligentes, com a capacidade de entender o significado das informações, e assim, recuperar e manipular de maneira lógica os dados. Para isso, é necessário que na implementação e estruturação de conteúdo, os provedores das páginas \textit{web} organizem suas informações seguindo regras de inferência, possibilitando gerir um raciocínio automatizado\cite{berners:2001}. Esta interligação de informações estruturadas, definidas semanticamente proporcionam uma recuperação de informação mais eficaz, formando uma rede conectada por meio de ferramentas tecnológicas\cite{alves:2004}

\begin{quote}
A \textit{Web Semântica} é uma rede de informações interligadas de tal modo que possa ser facilemtne processada por máquinas, em escala global\cite{palmer:2009}.
\end{quote}

A \textit{Web Semântica} é composta por três elementos, sendo eles:
\begin{enumerate}
\item Representação do conhecimento - Gera uma estrutura ao conteúdo significativo de páginas \textit{web} criando um ambiente onde agentes inteligente buscam de uma página a outra, podendo executar taréfas mais sofisticadas para usuários;
\item Ontologias -  Considerada a espinha dorsal da \textit{Web Semântica} define um aspecto semântico para representar seres, entes, qualquer que possa ser conveniente se chamar de assunto, conteúdos temáticos dos registros da realidade;
\item Agentes - Tem a função de coletar conteúdo e informações na \textit{web} em diversas fontes, processar e trocar dados entre outros programas, páginas ou aplicações através de linguagem que expressa inferências lógicas, resultado do uso das regras e informações como aquelas especificadas pela ontologia. Assim a máquina passa a reconhecer provas escritas na linguagem, estas quais os programas-agente, através da lógica descrita na ontologia, retorna dados requisitados pela pesquisa, respeitando o contexto.
\end{enumerate}

